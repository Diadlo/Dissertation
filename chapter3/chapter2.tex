\chapter{Реализация}

\section{Средства реализации}

\begin{itemize}
	\item Язык программирования С для написания внедряемой библиотеки;
	\item Язык программирования Python для написания графического интерфейса и
		генератора кода;
	\item Система сборки CMake.
\end{itemize}

В данной работе используются два языка программирования по следующим причинам:

\begin{itemize}
	\item Язык С предоставляет низкоуровневый интерфейс, который позволяет
		переопределить нужные функции во внедряемой библиотеке и в то же время
		без особой сложности в нем можно реализовать передачу данных в сокет.
	\item Язык Python, напротив, позволяет писать высокоуровневый код, что
		упрощает разработку прикладных приложений (в частности приложений с
		графическим интерфейсом).
\end{itemize}

\section{Требования к программному и аппаратному обеспечению}

Приложение предназначено для использования на IBM PC-совместимых компьютерах с
операционной системой Linux.

Работы приложения требуются:

\begin{itemize}
	\item аппаратное обеспечение согласно требованиям ОС;
	\item наличие интерпритатора Python;
\end{itemize}

Требования к целевому приложению:

\begin{itemize}
	\item приложение должно использовать графическую библиотеку Qt;
	\item должна быть произведена динамическая линковка с данной библиотекой.
\end{itemize}
