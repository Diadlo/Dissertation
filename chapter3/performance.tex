\section{Анализ производительности}

\subsection{Синтетическое тестирование}

Синтетическим тестированием называют запуск стандартизированных тестов,
на которых производятся замеры скорости работы тестируемой программы с целью
получения объективных критериев показывающих производительность системы.

В данной работе было разработано средство для добавления дополнительной
функциональности в существующее приложение. Из этого следует, что мы можем
поменять уже сущестовавшее поведение приложения. Из-за того, что новая
библиотека сканирует все объекты для определения их иерархической структуры, оно
добавляет функцию сложности $O(N)$ для действий интерфейса, а из этого следует,
что при увеличении числа графических элементов, производительность будет
деградировать.

Для произведения замеров было разработанно тестовое приложение. Оно принимает в
качестве параметра число окон, которые должно открыть. Каждое окно состоит
прямоугольника с $N \times M$ кнопками, где значения $N$ и $M$ заданы на этапе
сборки. Это приложение при запуске замеряет среднее время, которое требуется
для открытия одного окна.

На рис.~\ref{normal-perf} можно видеть, что в нормальном режиме работы
время открытия окна примерно постоянно и составляет $5.6\pm1$ cекунду.

\begin{figure}[h]
	\centering
	\begin{filecontents*}{normal.csv}
	num,timeout
	1,6
	2,4.5
	3,6.33333
	4,5
	5,4.4
	6,5.5
	7,5.42857
	8,6.25
	9,5.77778
	10,5.9
	11,5.36364
	12,5.5
	13,6
	14,6.28571
	15,6.53333
	16,5.6875
	17,5.70588
	18,5.66667
	19,5.36842
	20,5.85
	21,6.14286
	22,5.13636
	23,5.26087
	24,6.16667
	25,5.64
	26,5.5
	27,5.62963
	28,5.03571
	29,5.34483
	30,5.93333
	31,5.12903
	32,5.71875
	33,6.48485
	34,5.76471
	35,4.97143
	36,5.33333
	37,5.59459
	38,5.81579
	39,5.48718
	40,5.9
	41,5.4878
	42,5.66667
	43,5.65116
	44,5.43182
	45,5.6
	46,5.82609
	47,5.06383
	48,5.45833
	49,5.53061
	50,5.36
\end{filecontents*}
\begin{tikzpicture}
	\begin{axis}[height=6.5cm, width=9cm,
				 grid=both,
				 legend pos=outer north east,
				 xlabel=Число окон, ylabel=Время открытия окна (мс)]
		\addlegendentry{Время открытия}
		\addplot table [x=num, y=timeout, col sep=comma] {normal.csv};
		\addlegendentry{Среднее значение}
		\addplot[color=red, line width=2pt] coordinates {(0,5.6) (50,5.6)};
	\end{axis}
\end{tikzpicture}

	\caption{Скорость открытия окна без библиотеки}\label{normal-perf}
\end{figure}

На рис.~\ref{lib-perf} можно видеть, что при запуске приложения с использованием
библиотеки для сбора информации, среднее время открытия одного окна начинает
линейно расти при увеличении числа элементов (окон и кнопок в этих окнах).

\begin{figure}[h]
	\centering
	\begin{tikzpicture}
	\begin{filecontents*}{injected.csv}
		num,timeout
		1,28
		10,162.818
		20,305.143
		30,448.323
		40,610.171
		50,779.235
	\end{filecontents*}

	\begin{axis}[height=6.5cm, width=9cm,
				 grid=both,
				 legend pos=outer north east,
				 xlabel=Число окон, ylabel=Время открытия окна (мс)]
		\addlegendentry{Время открытия}
		\addplot table [x=num, y=timeout, col sep=comma] {injected.csv};
	\end{axis}
\end{tikzpicture}

	\caption{Скорость открытия окна с библиотекой}\label{lib-perf}
\end{figure}

\subsection{Ручное тестирование}

Кроме синтетического производилось и ручное тестирование в прикладных
приложениях. Для этого через приложение управления запускалась прикладаня
программа и в ней производились активация хотя бы одной кнопки и одного из
элементов главного меню.  Во время такой проверки падение производительности не
ощущалось.
