\section{Анализ производительности}

\subsection{Синтетическое тестирование}

Синтетическим тестированием называют запуск стандартизированных тестов,
на которых производятся замеры скорости работы тестируемой программы с целью
получения объективных критериев показывающих производительность системы.

В данной работе было разработано средство для добавления дополнительной
функциональности в существующее приложение. Из этого следует, что мы можем
поменять уже сущестовавшее поведение приложения. Из-за того, что новая
библиотека сканирует все объекты для определения их иерархической структуры, оно
добавляет функцию сложности $O(N)$ для действий интерфейса, а из этого следует,
что при увеличении числа графических элементов, производительность будет
деградировать.

Для произведения замеров было разработанно тестовое приложение. Оно принимает в
качестве параметра число окон, которые должно открыть. Каждое окно состоит
прямоугольника с $N \times M$ кнопками, где значения $N$ и $M$ заданы на этапе
сборки. Это приложение при запуске замеряет среднее время, которое требуется
для открытия одного окна.

На рис.~\ref{normal-perf} можно видеть, что в нормальном режиме работы
время открытия окна примерно постоянно и составляет $5.6\pm1$ cекунду.

\begin{figure}[h]
	\centering
	\input{schemes/normal-perf}
	\caption{Скорость открытия окна без библиотеки}\label{normal-perf}
\end{figure}

На рис.~\ref{lib-perf} можно видеть, что при запуске приложения с использованием
библиотеки для сбора информации, среднее время открытия одного окна начинает
линейно расти при увеличении числа элементов (окон и кнопок в этих окнах).

\begin{figure}[h]
	\centering
	\begin{tikzpicture}
	\begin{filecontents*}{injected.csv}
		num,timeout
		1,28
		10,162.818
		20,305.143
		30,448.323
		40,610.171
		50,779.235
	\end{filecontents*}

	\begin{axis}[height=6.5cm, width=9cm,
				 grid=both,
				 legend pos=outer north east,
				 xlabel=Число окон, ylabel=Время открытия окна (мс)]
		\addlegendentry{Время открытия}
		\addplot table [x=num, y=timeout, col sep=comma] {injected.csv};
	\end{axis}
\end{tikzpicture}

	\caption{Скорость открытия окна с библиотекой}\label{lib-perf}
\end{figure}

\subsection{Ручное тестирование}

Кроме синтетического производилось и ручное тестирование в прикладных
приложениях. Для этого через программу управления запускалось прикладное
приложение и в нем производились активация хотя бы одной кнопки и одного из
элементов главного меню. Во время такой проверки падение производительности не
ощущалось.
