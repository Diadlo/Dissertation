\section{Средства реализации}

\begin{itemize}
	\item Язык программирования С для написания внедряемой библиотеки;
	\item Язык программирования Python версии 3.8.3 для написания графического
		интерфейса и генератора кода;
	\item Система сборки CMake версии 3.2.
	\item Система управления версиями Git.
\end{itemize}

В данной работе используются два языка программирования по следующим причинам:

\begin{itemize}
	\item Язык С предоставляет низкоуровневый интерфейс, который позволяет
		переопределить нужные функции во внедряемой библиотеке и в то же время
		без особой сложности в нем можно реализовать передачу данных в сокет.
	\item Язык Python, напротив, позволяет писать высокоуровневый код, что
		упрощает разработку прикладных приложений (в частности приложений с
		графическим интерфейсом).
\end{itemize}

Во время разработки приложения управления на языке Python, с целью избежать
дублирования кода, были использованы готовые внешние модули:

\begin{itemize}
	\item \textbf{Qt} для реализации графического интерфейса;
	\item \textbf{xdk} для разбора файлов стандарта freedesktop;
	\item \textbf{system\_hotkey} для назначения гарячих клавиш на уровне
		операционной системы;
	\item \textbf{ewmh} для взаимодействия с графической подсистемой X11;
	\item \textbf{rofi} для отображения палитры команд и выполнения нечеткого
		поиска;
	\item \textbf{subprocess} для запуска внешних приложений.
\end{itemize}

Все модули доступны для установки с помощью Pip — официального пакетного
менеджера Python для Linux.

