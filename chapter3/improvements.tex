\section{Возможные усовершенствования}

После реализации базового функционала осталось место для для дальнейших
исследований и улучшений как в плане расширения функционала так и улучшения
существующего.

В рамках выполненной работы была добавлена регистрация простых и наиболее
распространенных элементов ГИП (кнопки, флажки, пункты меню). Однако для
удобства пользователя стоит расширить реализацию и добавить регистрацию более
сложных элементов (вкладки, таблицы), а также добавить поддержку универсальных
команд для работы с окнами: закрыть, минимизировать, свернуть.

Ранее я рассматривал \cite{polshakovvoice} возможность получения дополнительной
информации о графических элементах управления. При использовании Qt разработчик
может указывать описание к элементу, которое предназначено для людей с
ограниченными возможностями. Эти данные могут быть использованы при поиске
команды в палитре.

Также стоит рассмотреть возможность расширения числа перехватываемых функций,
чтобы избавится от использования алгоритмический не оптимальной функции для
задачи установления иерархии.

Как указывалось ранее, в палитре команд иногда указывают сочетание горячих
клавиш, если оно привязано к команде. Qt предоставляет штатное средство для
регистрации горячих клавиш — класс \code{QShortcut}. В случае, когда разработчик
корректно настроил горячие сочетания для действий \code{QAction}, библиотека в
состоянии установить связь между ними.

Система сигналов и слотов — это механизм коммуникации между объектами
интерфейса, используемый в Qt. Сигнал срабатывает в момент определенного
события (нажатия кнопки, клик мыши), а слот – это функция, которая вызывается
в ответ на определенный сигнал. Преимуществом такого подхода является
типобезопасность и слабосвязанность: класс, вырабатывающий сигнал ничего не
знает о том, какие слоты его получат.

В библиотеку может быть добавлена логика для анализа связей сигналов и слотов,
чтобы находить дополнительные связи: если в качестве источника сигнала выступает
\code{QShortcut}, а в качестве приемника — известный элемент интерфейса, то
можно связать грячие клавиши с активацией элемента.

В ОС Linux последнее время начинает набирать популярность графическая система
Wayland как замена X11\cite{wayland}. Qt уже адаптирован для работы с ним.
Однако решение, представленное в данной работе опирается на механизмы Х11 для
инициации действия внутри стороннего приложения. Стоит рассмотреть возможность
использования более универсального механизма "--- POSIX сигналы. В системах
совместимых с POSIX, существуют специальные сигналы, которые пользователь может
переопределить для своих целей. Они используются достаточно редко графическими
приложениями, поэтому библиотека может попытаться зарегистрировать свой обработчик
сигнала, а затем сервер будет посылать этот сигнал.
