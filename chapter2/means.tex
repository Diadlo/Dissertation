\section{Средства реализации}

Для разработки комплекса программ использовались следующие технологии и
инструменты:

\begin{itemize}
	\item Языки программирования С/С++ для написания внедряемой библиотеки;
	\item Язык программирования Python версии 3.8.3 для написания графического
		интерфейса и генератора кода;
	\item Система сборки CMake версии 3.2;
	\item Система управления версиями Git.
\end{itemize}

\subsection{Выбор языка}

В данной работе используются два языка программирования. Каждый из них применяется
в специфичной для него области.

Язык С предоставляет низкоуровневый интерфейс, который позволяет
переопределить нужные функции во внедряемой библиотеке и в то же время
без особой сложности в нем можно реализовать передачу данных в сокет.
Для работы с ассоциативными массивами и реализации перегрузки функцией в
некоторых местах был использован язык С++.

Язык Python, напротив, позволяет писать высокоуровневый код, что
упрощает разработку прикладных приложений (в частности с графическим
интерфейсом.

\subsection{Используемые модули Python}

Во время разработки приложения управления на языке Python, с целью избежать
дублирования кода, были использованы существующие внешние модули:

\begin{itemize}
	\item \textbf{Qt} для реализации отображения иконки в панели уведомлений
		и для реализации многопоточности при работе сервера;
	\item \textbf{xdk} для разбора файлов стандарта freedesktop, которые нужны
		для получения списка приложений, которые пользователь может запустить;
	\item \textbf{system\_hotkey} для назначения горячих клавиш на уровне
		операционной системы, т.к. средства Qt позволяют назначать их только
		в рамках самого приложения;
	\item \textbf{ewmh} для создания и отправки сообщения об активации окна для
		графической подсистемы X11;
	\item \textbf{rofi} для отображения палитры команд поверх всех окон и
		выполнения нечеткого поиска среди всех доступных команд;
	\item \textbf{subprocess} для запуска и последующего управления внешними
		программами в задаваемом окружении.
\end{itemize}

Все модули доступны для установки с помощью Pip — официального пакетного
менеджера Python для Linux.

