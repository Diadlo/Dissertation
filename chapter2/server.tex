\section{Реализация сервера}

Для реализации функционала сервера был реализован одноименный класс. Вся
основная работа данного класса происходит в отдельном потоке, которым он сам и
управляет. При инициализации класс создает файловый сокет и поток, для
последующего исполнения.

Непосредственно для своей работы класс \code{Server}, использует вспомогательные
классы:
\begin{itemize}
    \item \code{WidgetInfo} — описание одного элемента графического интерфейса.
        Состоит из:
        \begin{itemize}
            \item \code{addr} — идентификатор, который указан в протоколе. На
                практике используется адресс объекта внутри приложения;
            \item \code{text} — текст, соответствующий элементу (надпись на
                кнопке, название пункта меню).
        \end{itemize}

    \item \code{Window} — описание одного окна приложения. Состоит из
        \begin{itemize}
        \item \code{wid} — идентификатор окна, используемый для работы с X11;
        \item \code{widgets} — список всех элементов в этом окне.
        \end{itemize}

    \item \code{App} — описание приложения. Состоит из
        \begin{itemize}
        \item \code{pid} — идентификатор, который указан в протоколе; На
            практике используется системный идентификатор процесса;
        \item \code{client} — сокет, в который пишет сервер, для связи с
            приложением;
        \item \code{windows} — список окон приложения;
        \end{itemize}
\end{itemize}

Класс \code{Server} имеет следующие поля:

\begin{itemize}
    \item \code{applications} — список всех подключенных приложений (см.
        \code{App} выше);
    \item \code{last\_window\_id} — идентификатор последнего активного окна.
        Используется для возврата фокуса после отображения палитры команд;
    \item \code{socket} — сокет сервера, в который пишут клиенты;
    \item \code{running} — флаг, сигнаизирующий о том, должен ли сервер
        прекращать работу;
    \item \code{thread} — объект для управления потоком.
\end{itemize}

Все вышеперечисленные поля являются доступными только для данного класса.
Поэтому в коде их имена начинаются с двух символов подчеркивания.

Класс \code{Server} имеет следующие методы:

\begin{itemize}
    \item \code{start} — функция запуска потока;
    \item \code{stop} — функция остановки выполнения основной функции потока.
        Выставляет флаг \code{running} в значение \code{false}, чтобы следующая
        итерация не запустилась;
    \item \code{loop} — основной цикл потока. Ожидает данные из сокета и при
        их получении вызывает функции для обработки команд;
    \item \code{handle\_cmd} — функция обработки команд от клиентов.
        Занимается разбором данных, пришедших из сокета. На основе этих данных
        выбирает функцию обработчика конкретной команды;
    \item \code{add\_new\_app} — функция обработчик команды «добавить
        приложение». Из полученных данных создает объект \code{App} и добавляет
        его в информационную базу;
    \item \code{set\_widget\_text} — функция обработчик команды «установить
        текст элемента». Проверяет наличие элемента по идентификатору. Если
        элемент не найден, создает новый, иначе обновляет уже существующий;
    \item \code{set\_widget\_window} — функция обработчик команды «привязка
        элемента к окну». Удаляет из старого окна, добавляет в новое. Если
        нового нет — создает;
    \item \code{remove} — функция-обработчик команды «удалить элемент».
        Находит окно с элементом и удаляет из него информацию о нем.
    \item \code{activated} — функция обработчик сообщения «об активации окна».
        Сохраняет пришедший идентификатор окна в поле \code{last\_window\_id}
        для дальнейшего использования;
    \item \code{activate} — функция, выполняющая поиск элемента по его
        названию и вызывающая функцию активации элемента;
    \item \code{activate\_widget} — функция активации элемента. Посылает
        команду клиенту и активирует окно через Х11 для передачи события;
    \item \code{get\_options} — функция получения всех доступных вариантов для
        активации (список строк);
    \item \code{get\_app} — вспомогательная функция для получения приложения
        по идентификатору процесса;
    \item \code{find\_window\_by\_wid} — вспомогательная функция для получения
        окна по идентификатору окна.
\end{itemize}

Также есть ряд функций, которые не входят в класс, но используются им:
\begin{itemize}
    \item \code{recv\_uint32}, \code{recv\_uint64} — функции чтения из сокета
        4-х и 8-и байт соответственно;
    \item \code{recv\_text} — функция получения строки, описываемой протоколом;
    \item \code{activate\_widget} — функция, посылающая в сокет клиента команду,
        на активацию элемента;
    \item \code{activate\_window} — функция, посылающая графической подсистеме
        X11 команду на активацию окна.
\end{itemize}
