\chapter{Реализация}

\section{Средства реализации}

\begin{itemize}
	\item Язык программирования С для написания внедряемой библиотеки;
	\item Язык программирования Python версии 3.8.3 для написания графического
		интерфейса и генератора кода;
	\item Система сборки CMake версии 3.2.
	\item Система управления версиями Git.
\end{itemize}

В данной работе используются два языка программирования по следующим причинам:

\begin{itemize}
	\item Язык С предоставляет низкоуровневый интерфейс, который позволяет
		переопределить нужные функции во внедряемой библиотеке и в то же время
		без особой сложности в нем можно реализовать передачу данных в сокет.
	\item Язык Python, напротив, позволяет писать высокоуровневый код, что
		упрощает разработку прикладных приложений (в частности приложений с
		графическим интерфейсом).
\end{itemize}

\section{Требования к программному и аппаратному обеспечению}

Приложение предназначено для использования на IBM PC-совместимых компьютерах с
операционной системой Linux.

Работы приложения требуются:

\begin{itemize}
	\item аппаратное обеспечение согласно требованиям ОС;
	\item ОЗУ не менее 1 ГБ;
	\item 1 Гб свободного места;
	\item наличие интерпритатора Python версии не ниже 3.0.
\end{itemize}

Требования к целевому приложению:

\begin{itemize}
	\item приложение должно использовать графическую библиотеку Qt;
	\item должна быть произведена динамическая линковка с данной библиотекой.
\end{itemize}

\section{Генератор кода}

Файл, который подается на вход генератору должен представлять собой готовый к
использованию исходный код, за исключением функций-оберток. Функции обертки
должны быть сопоставлены с функциями-обработчиками. Для добавления возможности
такого сопоставления было принято решение добавить специальные комментарии,
которые состоят из:

\begin{enumerate}
	\item Опорного элемента (\code{//method}). Он требуется для того, чтобы
		отличать описательные комментарии от обычных.
	\item Сигнатура метода в которой не указываются имена переменных.
	\item Разделитель сигнатуры и названия функции-обработчика. В качестве
		разделителя выступает стрелка \code{->}.
	\item Название функции-обработчика.
\end{enumerate}

Пример комментария: \code{//method bool QWidget::event(QEvent*) -> checkEvent}

Вместо каждого такого комментария генератор подставляет код функции, реализующий
следующий алгоритм:

\begin{algorithm}[H]
  \Fn{методБиблиотеки({$a$, $b$, $c$})}{
    \If{реальнаяФункция == NULL}{
      реальнаяФункция = следующийСимвол("методБиблиотеки")\;
    }

    \If{инициализацияУспешна()}{
      функцияОбработчик($a$, $b$, $c$)\;
    }
    \KwRet\ реальнаяФункция($a$, $b$, $c$);
  }
  %\caption{Общий вид алгоритма подмены функции}
\end{algorithm}

Но если бы этого было достаточно, то для подставновки можно было использовать
макросы из языка Си. Однако сложность заключается в получении имени функции для
подмены. Это связано с тем, что символы в исполняемом файле должны представлять
собой просто идентификатор, который не предусматривает сам по себе типы
параметров, возвращаемых значений, имена классов и т.п.
% Тут можно добавить про то, почему каждый из этих пунктов нужен.
Однако вся эта информация должна быть сохранена. Стандарт языка С++ не говорит,
как именно должно происходить данное преобразование, поэтому это будет зависить
от реализации. Однако существуют стандарты ABI (application binary interface,
двоичный интерфейс приложения), которые определяют как должно происходить данное
преобразование. Это позволяет библиотекам, собраным разными компиляторами с
одним стандартом ABI, работать друг с другом. Для Linux одним из неиболее
распространенных стандартов ABI является Itanium\cite{itaniumabi}

Т.к. в рамках данной работы не нужно генерировать имена с использованием
пространств имен, виртуальных таблиц и т.п., была реализована только часть
стандарта, описывающая получение имени функции и некоторых типов.

\begin{minted}{text}
<имя-функции> ::= _Z <имя>
<имя> ::= <вложенное-имя> <параметры-функции>
<вложенное-имя> ::= N <префикс> <явное-имя> E
<префикс> ::= <префикс> <явное-имя>
          ::= <явное-имя>
<явное-имя> ::= <имя-из-исходного-кода>
            ::= <конструктор-или-деструктор>
<имя-из-исходного-кода> ::= <длина-идентифкатора> <идентификатор>
<конструктор-или-деструктор> ::= С1
                             ::= D1

<параметры-функции> ::= <тип>
                    ::= <тип> <параметры-функции>
<тип> ::= <встроенный-тип>
      ::= <квалифицированный-тип>
      ::= <имя-класса-или-перечисления>

<встроенный-тип> ::= v # void
                 ::= i # int
                 ::= c # char
                 ::= b # bool

<квалифицированный-тип> ::= <квалификатор> <тип>
<квалификатор> ::= K # const
               ::= R # reference

<имя-класса-или-перечисления> ::= <имя>
\end{minted}


\section{Протокол}

Клиент и сервер общаются с помощью специального протокола. В рамках протокола
передаются команды.

\begin{minted}{text}
<команда> ::= <имя-команды> <параметры-команды>
<имя-команды> ::= <строка>
<стока> ::= <длина-строки> <идентификатор>
<длина-строки> ::= uint32_t
<идентификатор> ::= "newApp"
                ::= "setWidgetText"
                ::= "remove"
                ::= "activated"
                ::= "setWidgetWindow"
                ::= "activate"
\end{minted}

Рассмотрим существующие команды:

\subsection{Регистрация приложения}

\textbf{Идентификатор:} «newApp».

\textbf{Праметры:}
\begin{enumerate}
\item идентификатор процесса (8 байт);
\item адрес сокета для общения (строка).
\end{enumerate}

Данная команда посылается от клиента к серверу в самом начале. Серверу
необходимо дифференцировать различные приложения, чтобы корректно посылать
команды.

\subsection{Установка текста элемента}

\textbf{Идентификатор:} «setWidgetText».

\textbf{Праметры:}
\begin{enumerate}
\item идентификатор процесса (8 байт);
\item идентификатор элемента (8 байт);
\item текст элемента (строка).
\end{enumerate}

С помощью данной команды клиент сообщает серверу о добавлении или изменении
текста элемента. Если идентификатор уже использовался, то считается, что текст
элемента изменился, иначе, что элемент только добавлен.

\subsection{Удаление элемента}

\textbf{Идентификатор:} «remove».

\textbf{Праметры:}
\begin{enumerate}
\item идентификатор процесса (8 байт);
\item идентификатор элемента (8 байт).
\end{enumerate}

Используется, когда клиент считает, что элемент больше не доступен для
активации. После выполнения данной команды, сервер должен перестать выдавать
элемент в списке с запросом всех доступных элементов. Идентификатор считается
свободным и может быть повторно использован для новых элементов.

\subsection{Сообщение об активации окна}

\textbf{Идентификатор:} «activated».

\textbf{Праметры:}
\begin{enumerate}
\item идентификатор процесса (8 байт);
\item идентификатор окна (4 байта).
\end{enumerate}

Клиент посылает данную команду, когда сменилось активное окно в приложении. Это
требуется для того, чтобы сервер мог позже вернуться к последнему активному
окну. Идентификатор окна представляет собой аттрибут \textit{windowId} окна
системы X11. Сервер должен использовать его для последнующей работы с окном.

\subsection{Привязка элемента к окну}

\textbf{Идентификатор:} «setWidgetWindow».

\textbf{Праметры:}
\begin{enumerate}
\item идентификатор процесса (8 байт);
\item идентификатор элемента (8 байт);
\item идентификатор окна (4 байта).
\end{enumerate}

Некоторые приложения могут создавать элементы, которые не привязаны к
конкретному окну, а привязывать их позже. Или перемещать существующие элементы
между окнами в целях оптимизации. Клиент должен посылать данную команду, чтобы
сообщить серверу, что конкретный элемент относится к конкретному окну. Серверу
это нужно, чтобы понимать, какие элементы доступны в текущем окне.

\subsection{Активация элемента}

\textbf{Идентификатор:} «activate».

\textbf{Праметры:}
\begin{enumerate}
\item идентификатор элемента (8 байт);
\end{enumerate}

Эта команда посылается от сервера клиенту, чтобы активировать элемент (нажать
кнопку, открыть меню и т.п.)

\subsection{Обработка ошибок}

В случае, если в любой команде, которая использует идентификатор элемента,
приходит значение, которое не было зарегистрировано или было удално, сервер
должен проигнорировать команду.

\section{Реализация библиотеки для внедрения}

Библиотека, которая будет использоваться для сбора информации, должна
реализовывать часть функций аналогично библиотеке Qt и выполнять следующие
функции:

\begin{itemize}
    \item выполнять регистрацию приложения;
    \item фиксировать создание объектов (в Qt элементы ГИП называюся
        виджетами);
    \item фиксировать смену описания объекта;
    \item фиксировать перемещение элементов между окнами;
    \item фиксировать удаление объектов;
    \item активировать элемент по команде.
\end{itemize}

\subsection{Регистрация приложения}

Регистрация приложения должна происходить до создания первого объекта, но
универсального метода, который вызывался бы в самом начале и был бы всегда
доступен для переопределения, к сожалнию, нет. Поэтому каждый переопределяемый
метод в начале проверяет, была ли проведена инициализация. И если нет, то
проводит ее, подключаясь к серверу, создавая сокет для получения команд от
сервера и посылая на сервер команду регистрации текущего приложения.

\subsection{Создание объекта}

В рамках данной работы была реализована работа регистрация создания чекбоксов
(\code{QCheckBox}), кнопок (\code{QPushButton}) и «действий» (\code{QAction}).
«Действия» в Qt это абстракция именованной команды, которая может быть
использована в ГИП\cite{qaction}. Они используются в главном меню, в контекстном
меню, в качестве обработчиков горячих клавиш и т.п.

Создание объектов всегда происходит через вызов конструкторов классов. Не все
конструкторы позволяют указывать текст описания элемента. Поэтому для реализации
более общего случая конструкторы с указанием текста рассматриваются как
выполнение двух действий: создание объекта и установка текста.

\subsection{Смена описания}

Почти все виджеты в Qt позволяют менять связанный с ними текст (надпись на
кнопке, название пункта меню и т.п.) во время исполнения. Для всех указанных
выше типов был переопределен метод \code{setText}. При вызове этого метода,
клиент посылает на сервер команду установки текста элемента.

\subsection{Перемещение элементов между окнами}

Qt не имеет доступного для переопределения метода, который позволял бы
отслеживать перемещения объекта. Поэтому в библиотеке пришлось переопределить
метод \code{QWidget::event}. Он вызывается при всех событиях, на которые должен
отреагировать хотя бы один виджет. А фиксация изменения окна у виджета, происходит
в два этапа:

\begin{enumerate}
  \item Получить список всех окон.
  \item Для каждого окна проверить все его дочерние элементы.
\end{enumerate}

Qt использует отличные от X11 понятия «окна». Так, например, элемент может быть
отображен без явного создания соответствующего окна. Тогда на уровне Qt будет
просто виджет, а на уровне X11 --- окно. Поэтому отдельной задачей является
получение списка всех окон в терминах X11. Таковыми являются все виджеты
верхнего уровня, у которых нет родителя.

%\begin{algorithm}[H]
%  окна = пустойСписок()
%  \ForEach{виджет}{Приложение::виджетыВерхнегоУровня()}{
%    \If{виджет.родитель == NULL}{
%      \Continue
%    }
%
%    \If{окна.содержит(виджет)}{
%      \Continue
%    }
%
%    окна.добавить(виджет)\;
%  }
%  %\caption{Определение смены окна}
%\end{algorithm}

Затем запускается рекурсивный алгоритм, который проверяет, есть ли среди
дочерних элементов новые.

\begin{algorithm}[H]
  \Fn{привязатьЭлементКОкну(элемент, новоеОкно)}{
    староеОкно = окноЭлемента[элемент]\;
    \If{староеОкно == NULL или староеОкно != новоеОкно}{
      послатьКомандуСменыОкна(элемент, новоеОкно)\;
      окноЭлемента[элемент] = новоеОкно\;
    }

    \ForEach{дочернийЭлемент в элемент.подэлементы()}{
      привязатьЭлементКОкну(дочернийЭлемент, новоеОкно)\;
    }
  }
  %\caption{Рекурсивное обновление элементов окна}
\end{algorithm}

Такой метод не слишком оптимален. В рамках улучшения данного решения стоит
поискать возможность усовершенствовать его.

\subsection{Удаление элементов}

В текущей реализации элемент считается недоступным после его удаления из памяти
(т.е. при вызове деструктора), но для удобства работы правильнее было бы
обрабатывать события скрытия и отображения элемента.

\subsection{Активация по команде}

Как было указано в моей прошлой работе\cite{polshakovinject}, библиотека не
может выступать инициатором действия, она может только реагировать на события.
Поэтому для активации элемента требуется событие, которое, например, может
послать сервер. В текущей реализации была выбрана функция обработки событий как
наиболее часто вызываемая. Именно в ней проверяется наличие команды активации.

