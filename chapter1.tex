\chapter{Обзор существующих реализаций палитр команд}

Впервые палитра команд появилась 1 июля 2011 году в редакторе Sublime Text
2~\cite{sublimetext2changelog}. Вслед за этим подобный функционал был реализован
в некоторых других программах. Таких как:
\begin{itemize}
	\item Atom\cite{atom},
	\item VSCode\cite{vscode},
	\item JupyterLab\cite{jupyterlab}.
\end{itemize}

Но это были лишь единичные случаи. В апреле 2017 года появилась альфа версия
приложения Plotinus\cite{plotinus}, которое позволяет добавлять палитру команд в
любое приложение, которое написано с использованием графической библиотеки GTK.

\begin{figure}
	\includegraphics[width=\textwidth]{vscode}
	\caption{VSCode}
\end{figure}

\begin{figure}
	\includegraphics[width=\textwidth]{SublimeText}
	\caption{Sublime Text}
\end{figure}

\begin{figure}
	\includegraphics[width=\textwidth]{Plotinus}
	\caption{Plotinus}
\end{figure}

