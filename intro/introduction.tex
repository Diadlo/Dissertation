\chapter*{Введение}
\addcontentsline{toc}{chapter}{Введение}

С ростом возможностей персональных компьютеров растут и возможности программного
обеспечения. Такие группы программ как графические редакторы, текстовые
процессоры среды разработки и многое другое обрастают огромным числом функций.
Для доступа к этим функциям используются элементы ГИП.

Для быстрого доступа к функциям приложения, используются горячие клавиши. Но
обычно горячие клавиши создаются только для самых часто используемых команд,
остальные же приходится выбирать вручную в интерфейсе.

Но как бы хорошо ни был разработан интерфейс, число функций может оказаться
настолько большим, что появляется проблема с поиском нужного элемента
управления. Кроме того, некоторые системы поддерживают возможность добавления
сторонних модулей. В таком случае место расположения элементов регулируется
сторонними разработчиками, а не авторами оригинального приложения.

Для упрощения поиска элементов была придумана технология «палитра команд». Это
специальное окно в интерфейсе приложения, где отображаются все доступные
функции.
Иногда рядом с функцией отображается сочетание горячих клавиш, с помощью
которого эта функция может быть вызвана. Это помогает пользователю легко
запоминать новые сочетания, который он забыл или не знал раньше.
В этом же окне есть поле для ввода поискового запроса.

Такой функционал изначально начал появлятся в различных текстовых редакторах и
средах разработки. Но функционя оказалась достаточно удобной, поэтому была
разработана система, позволяющая добавлять такой функционал в приложения,
использующие фреймворк GTK.

Целью данной работы было реализовать механизм, который позволял бы добавлять
палитру команд в сторонние приложения без их пересборки. Из-за того, что такой
механизм для GTK уже реализован, был выбран другой (не меньший по
распространенности) фреймворк: Qt.

Для того, чтобы достичь поставленной цели нужно было исследовать возможность
добавления функционала в уже собранную программу, разработать библиотеку для
расширения произвольных приложений и программу для координации работы.

В первой главе данной работы рассмотрены существующие приложения, использующие
палитру команд, проведен анализ способов добавления функционала в существующее
приложение, сформулированы задачи для реализации и описана архитектура
взаимодействия всех элементов разрабатываемой системы.

Во второй главе рассмотрены детали и средства реализации каждого из элементов
системы, протокол взаимодействия их друг с другом, а также вспомогательные
средства, которые были разработаны для решения технических задач.

В третьей главе производится анализ полученного решения: сравнение
производительности приложения с использованием решения и без него,
рассматриваются возможности дальнейших улучшений.
