\chapter*{Введение}
\addcontentsline{toc}{chapter}{Введение}

С ростом возможностей персональных компьютеров растут и возможность программного
обеспечения. Такие группы программ как графические редакторы, текстовые
процессоры среды разработки и многое другое обрастают огромным числом функций.
Для доступа к этим функциям используются элементы ГИП.

Для быстрого доступа к функциям приложения, используются горячие клавиши. Но
обычно горячие клавиши создаются только для самых часто используемых команд,
остальные же приходится выбирать вручную в интерфейсе.

Но как бы хорошо ни был разработан интерфейс, число функций может оказаться
настолько большим, что появляется проблема с поиском нужного элемента
управления. Кроме того, некоторые системы поддерживают возможность добавления
сторонних модулей. В таком случае место расположения элементов регулируется
сторонними разработчиками, а не авторами оригинального приложения.

Для упрощения поиска элементов была придумана технология <<палитра команд>>. Это
специальное окно в интерфейсе приложения, где отображаются все доступные
функции.
Иногда рядом с функцией отображается сочетание горячих клавиш, с помощью
которого эта функция может быть вызвана. Это помогает пользователю легко
запоминать новые сочетания, который он забыл или не знал раньше.
В этом же окне есть поле для ввода поискового запроса.

Целью данной работы было реализовать механизм, который позволял бы добавлять
палитру команд в сторонние приложения без их пересборки.
