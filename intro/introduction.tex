\chapter*{Введение}
\addcontentsline{toc}{chapter}{Введение}

Из-за роста возможностей персональных компьютеров, программное обеспечение
становится все сложнее и функциональнее. Такие группы программ как графические
редакторы, браузеры, органайзеры и многое другое обрастают
огромным числом функций. Для доступа к ним используются элементы ГИП.

Для быстрого доступа к функциям приложения, используются горячие клавиши. Но
обычно горячие клавиши создаются только для самых часто используемых команд,
остальные же приходится выбирать вручную в интерфейсе.

Но как бы хорошо ни был разработан интерфейс, число функций может оказаться
настолько большим, что появляется проблема с поиском нужного элемента
управления. Кроме того, некоторые системы поддерживают возможность добавления
сторонних модулей. В таком случае место расположения элементов регулируется
сторонними разработчиками, а не авторами оригинального приложения.

Для упрощения поиска элементов была придумана технология, названная «палитра
команд». Это специальное окно в интерфейсе приложения, где отображаются все
доступные функции.
Иногда рядом с функцией отображается сочетание горячих клавиш, с помощью
которого она может быть вызвана. Такой подход помогает пользователю легко
запоминать новые сочетания, который он забыл или не знал раньше.
В этом же окне есть поле для ввода поискового запроса.

Такая функциональность начала появляться в различных текстовых редакторах и
средах разработки. Палитра команд оказалась достаточно удобной, поэтому позже
была разработана библиотека Plotinus, позволяющая добавлять такой функционал в
приложения, использующие фреймворк GTK. Данная библиотека опиралась на механизм,
предоставляемый самим GTK, для расширения готовых приложений. Более подробный
обзор приложений, использующих палитру команд выполнен в разделе \ref{analogs}.

Целью данной работы было реализовать механизм, который позволял бы добавлять
палитру команд в сторонние приложения без их пересборки. Из-за того, что такой
механизм для GTK уже реализован, был выбран другой (не меньший по
распространенности) фреймворк: Qt.

Для того, чтобы достичь поставленной цели нужно было исследовать возможность
добавления функционала в уже собранную программу, разработать библиотеку для
расширения произвольных приложений и программу для координации работы.

В первой главе данной работы рассмотрены существующие приложения, использующие
палитру команд, проведен анализ способов добавления функционала в существующее
приложение, сформулированы задачи для реализации и описана архитектура
взаимодействия всех элементов разрабатываемой системы.

Во второй главе рассмотрены детали и средства реализации каждого из элементов
системы, протокол взаимодействия их друг с другом, а также вспомогательные
средства, которые были разработаны для решения технических задач.

В третьей главе производится анализ полученного решения: сравнение
производительности приложения с использованием решения и без него,
рассматриваются возможности дальнейших улучшений.
