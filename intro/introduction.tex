\chapter*{Введение}
\addcontentsline{toc}{chapter}{Введение}

Из-за роста возможностей персональных компьютеров, программное обеспечение
становится все сложнее и функциональнее. Такие группы программ как графические
редакторы, браузеры, органайзеры и многое другое обрастают
огромным числом функций. Для доступа к ним используются элементы ГИП.

Для быстрого доступа к функциям приложения, используются горячие клавиши. Они
обычно создаются только для самых часто используемых команд. Остальные
же действия приходится производить вручную через интерфейс.

Как бы хорошо ни был разработан интерфейс, число функций может оказаться
настолько большим, что появляется проблема с поиском нужного элемента
управления. Кроме того, некоторые системы поддерживают возможность добавления
сторонних модулей. В таком случае место расположения элементов регулируется
сторонними разработчиками, а не авторами оригинального приложения.

«Палитра команд» — технология, которую придумали для упрощения поиска элементов.
Она представляет собой специальное окно в интерфейсе приложения, где
показываются все доступные команды. Иногда рядом с ними отображается
сочетание горячих клавиш, с помощью которых пользователь может её вызвать. Такой
подход помогает пользователю легко запоминать новые сочетания, который он забыл
или не знал раньше.  В этом же окне есть поле для ввода поискового запроса.

Такая функциональность появилась в различных текстовых редакторах и
средах разработки. Палитра команд оказалась достаточно удобной, поэтому позже
энтузиасты разработали библиотеку с открытым исходным кодом Plotinus. Она
позволяет добавлять такую функцию в приложения, которые используют фреймворк
GTK. Данная библиотека опиралась на механизм, который предоставляет сами GTK,
для расширения готовых приложений. Более подробный обзор решений,
которые используются палитру команд производится в разделе \ref{analogs}.

Целью данной работы было реализовать механизм, который позволял бы добавлять
палитру команд в сторонние приложения без их пересборки. Из-за того, что такой
механизм для GTK уже существует, был выбран другой (не меньший по
распространенности) фреймворк — Qt. Целевой ОС является Linux.

Для достижения поставленной цели, нужно было исследовать возможность
добавления функций в уже собранную программу, разработать библиотеку для
расширения произвольных приложений и программу для координации работы.

В первой главе данной работы рассматриваются существующие приложения, которые
используют палитру команд, проводится анализ способов добавления функций в
существующее приложение, формулируются задачи для реализации и описывается
архитектура взаимодействия всех элементов разрабатываемой системы.

Во второй главе рассматриваются детали и средства реализации каждого из элементов
системы, протокол взаимодействия их друг с другом, и вспомогательные
средства, которые были разработаны для решения технических задач.

В третьей главе производится анализ полученного решения: сравнение
производительности приложения с использованием решения и без него,
рассматриваются возможности дальнейших улучшений.
