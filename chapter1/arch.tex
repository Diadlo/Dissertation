\section{Архитектура системы}

Вся работа системы может быть рассмотрена в 3х частях:
\begin{enumerate}
    \item регистрация нового элемента в приложении;
    \item посылка команды;
    \item активация элемента.
\end{enumerate}

\begin{figure}
	\centering
	\begin{sequencediagram}
	\newthread{app}{\shortstack{Целевое\\приложение}}
	\newinst[1.3]{lib}{\shortstack{Подгружаемый\\модуль}}
	\newinst[1.3]{qt}{Qt}
	\newinst[2]{x11}{X11}
	\newthread{server}{Сервер}
	
	\postlevel
	\postlevel
	\begin{call}{app}{\shortstack{Перехват создания\\элемента}}{lib}{}
		\mess{lib}{Оповещение об изменении элемента}{server}
		
		\postlevel
		\postlevel
		\begin{call}{lib}{\shortstack{Оригинальный\\вызов}}{qt}{}
			\begin{call}{qt}{\shortstack{Отображение\\элемента}}{x11}{}
			\end{call}
		\end{call}
	\end{call}
\end{sequencediagram}

	\caption{Общий обзор архитектуры}\label{fig:arch}
\end{figure}

\begin{figure}
	\centering
	\begin{sequencediagram}
	\newthread{ctrl}{\shortstack{Приложение\\управления}}
	\newinst[2]{server}{Сервер}
	\newinst[2]{lib}{\shortstack{Подгружаемый\\модуль}}
	\newinst{x11}{X11}

	\postlevel
	\postlevel
	\begin{call}{ctrl}{\shortstack{Пользователь\\выбрал команду}}{server}{}
		\begin{messcall}{server}{\shortstack{Передача команды\\активации}}{lib}{}
		\end{messcall}
		\postlevel
		\begin{messcall}{server}{Активация окна}{x11}{}
		\end{messcall}
	\end{call}
\end{sequencediagram}

	\caption{Общий обзор архитектуры}\label{fig:arch}
\end{figure}

\begin{figure}
	\centering
	\begin{sequencediagram}
	\newthread{x11}{X11}
	\newinst[1.3]{qt}{Qt}
	\newinst[1.3]{lib}{\shortstack{Подгружаемый\\модуль}}
	\newinst{app}{\shortstack{Целевое\\приложение}}

	\postlevel
	\postlevel
	\begin{call}{x11}{\shortstack{Активация\\окна}}{qt}{}
		\begin{call}{qt}{\shortstack{Перехват\\события}}{lib}{}
			\begin{call}{lib}{\shortstack{Передача\\события}}{app}{}
			\end{call}
			\begin{call}{lib}{\shortstack{Инициация\\команды}}{qt}{}
				\postlevel
				\begin{call}{qt}{\shortstack{Обработка}}{qt}{}
					\postlevel
					\begin{call}{qt}{\shortstack{Выполнение\\команды}}{app}{}
					\end{call}
				\end{call}
			\end{call}
		\end{call}
	\end{call}
\end{sequencediagram}

	\caption{Общий обзор архитектуры}\label{fig:arch}
\end{figure}

Подходящая архитектура для такой задачи была предложена мной в одной из прошлых
работ\cite{polshakovinject}. На рис.~\ref{fig:arch} изображена схема
взаимодействия данного комплекса программ. На схеме изображены следующие
действия:
\begin{enumerate}
    \item Перехват функции создания элемента интерфейса
    \item Оповещение о создании элемента
    \item Вызов оригинальной функции графической библиотеки
    \item Отображение элемента управления
    \item Передача информации об элементе приложению управления
    \item Отображение палитры команд
    \item Выбор команды
    \item Вызов функции для выполнения команды
    \item Передача команды к выполнению
    \item Вызов функции активации окна целевого приложения
    \item Активация окна
    \item Перехват функции обработки события
    \item Вызов функции обработки события
    \item Вызов функции для выполнения команды
    \item Выполнение команды
\end{enumerate}

\begin{figure}
	\centering
	\begin{tikzpicture}[
        ->,>=stealth',node
        distance=2cm, semithick
    ]

    \tikzstyle{block} = [rectangle,draw,minimum height=0.8cm]

    \node[block] (preload) {Подгружаемый модуль};
    \node[block] (lib)    [above of=preload] {libQtWidgets};
    \node[block] (x11)    [above of=lib]     {X11};
    \node[block] (server) [left=2cm of preload] {Сервер управления};
    \node[block] (app)    [below of=preload] {Целевое приложение};

    \node[block] (ctrl) [above of=server] {Приложение управления};
    \node[block] (user) [above of=ctrl] {Пользователь};

    \path[]
        (app)     edge[bend left]     node[left]  {1} (preload)
        (preload) edge[bend left=5]   node[below] {2} (server)
        (preload) edge[bend left]     node[left]  {3} (lib)
        (lib)     edge[bend left]     node[left]  {4} (x11)
        (server)  edge[bend left]     node[left]  {5} (ctrl)
        (ctrl)    edge[bend left]     node[left]  {6} (user)
        (user)    edge[bend left]     node[right] {7} (ctrl)
        (ctrl)    edge[bend left]     node[right] {8} (server)
        (server)  edge[bend left=5]   node[above] {9} (preload)

        (server.north east)  edge    node[above]{10} (x11)

        (x11)     edge[bend left]     node[right] {11} (lib)
        (lib)     edge[bend left]     node[right] {12} (preload)
        (preload) edge[bend left]     node[right] {13} (app)
        (preload) edge[bend right=75] node[right] {14} (lib)
        (lib)     edge[bend left=80]  node[right]  {15} (app)
    ;

\end{tikzpicture}
 \\
	\caption{Общий обзор архитектуры}\label{fig:arch}
\end{figure}
