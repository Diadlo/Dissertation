\section{Построение кривой Безье по набору точек}

Прежде всего стоит отметить, что именно мы будем иметь в виду под кривой Безье. Дело в том, что в классическом
понимании у кривой Безье только первая производная непрерывна, а в наших задачах получаемая кривая имеет непрерывные
производные до произвольного порядка. Поэтому можно ввести дополнительное понятие \textit{улучшенной кривой Безье},
но для краткости мы будем называть её просто кривой Безье.

Характерными особенностями такой кривой является способ её построения и её внешний вид. В каждой из последующих задач
способ построения описан довольно подробно.

В каждой из задач данного раздела предполагается, что $k>2$, $n>0$ "--- некоторые произвольные наперёд заданные
натуральные числа.

\subsection*{На плоскости}

Рассмотрим плоскость действительных чисел $\mathbb{R}^2$. Пусть на ней задана последовательность различных точек $p_i$:

$$\{p_i: (x_i, y_i)\}, i \in \{1, \dots, k\}.$$

Пусть каждые две последовательные точки $p_i$ и $p_{i+1}$ соединены отрезками, и эти отрезки поделены пополам точками
$q_i, i \in \{1, \dots, k-1\}$.

Требуется построить параметризованную кривую $r(t) \in \mathbb{R}^2$, проходящую последовательно через точки $p_1,
q_1, q_2, \dots, q_{k-2}$, $q_{k-1}$, $p_k$ в направлении построенных отрезков. При этом данная кривая должна принадлежать
классу $C^n$, т.~е. иметь непрерывные производные в каждой точке до порядка $n$ включительно.

\subsection*{На двумерной сфере}

Рассмотрим двумерную сферу $S^2$, заданную в некоторой ортогональной системе координат с началом в центре этой сферы.
Пусть задана последовательность различных точек $p_i$, лежащих на поверхности этой сферы:

$$\{p_i\}, i \in \{1, \dots, k\}.$$

Пусть каждые две последовательные точки $p_i$ и $p_{i+1}$ соединены большими дугами, и эти дуги поделены пополам
точками $q_i, i \in \{1, \dots, k-1\}$.

Требуется построить параметризованную кривую $r(t)$, лежащую на поверхности сферы $S^2$ и проходящую последовательно
через точки $p_1, q_1, q_2, \dots, q_{k-2}$, $q_{k-1}$, $p_k$ в направлении построенных дуг. При этом данная кривая должна
принадлежать классу $C^n$, т.~е. иметь непрерывные производные в каждой точке до порядка $n$ включительно.

\subsection*{На ориентационной сфере}

Рассмотрим ориентационную сферу $S^3$, заданную в некоторой ортогональной сис\-теме координат с началом в центре этой
сферы. Пусть задана последовательность различных ориентаций $Q_i$, принадлежащих поверхности этой сферы:

$$\{Q_i\}, i \in \{1, \dots, k\}.$$

Пусть каждые две последовательные ориентации $Q_i$ и $Q_{i+1}$ соединены большими дугами, и эти дуги поделены пополам
ориентациями $N_i, i \in \{1, \dots, k-1\}$.

Требуется построить параметризованную кривую $R(t)$, лежащую на поверхности сферы $S^3$ и проходящую последовательно
через точки $Q_1, N_1, N_2, \dots, N_{k-2}$, $N_{k-1}$, $Q_k$ в направлении построенных дуг. При этом данная кривая должна
принадлежать классу $C^n$, т.~е. иметь непрерывные производные в каждой точке до порядка $n$ включительно.
